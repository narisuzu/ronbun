\begin{arigatou}
    能够完成本次毕业设计,首先要感谢开发过程中使用到的开源项目:

    Mozilla 的 Rust 程序设计语言;

    Facebook 的 GraphQL 和 React.js;

    jonobr1 的 Two.js:面向现代浏览器的二维绘图api;

    Apollo 的 GraphQL 网关和客户端实现;

    Microsoft 的 TypeScript 程序设计语言;

    PostgreSQL Global Development Group的PostgreSQL数据库管理系统;

    Tokio:Rust异步运行时;

    petgraph:Rust 图论算数据结构库;

    Actix Team 的 actix-web:高性能的异步Rust web框架;

    sunli829 的 Async-GraphQL:Rust的异步 GraphQL 服务端实现;

    Diesel Core Team 的 diesel:Rust 的SQL数据库ORM库;

    sfackler 的 r2d2:Rust 的通用数据库连接池;

    OpenJS 基金会的 Node.js:前端web服务器的运行时环境;

    Keats 的 rust-bcrypt:Rust的bcrypt加密算法库;

    Material-UI 的控制台模板。

    若无上述开源项目,本项目便不会存在。

    在使用上述项目时偶尔会遇到一些问题、但作者都会十分热心的解答我的问题,修复项目的bug。
    我由衷的感谢他们。另外感谢\hologo{LuaLaTeX} 、\CTeX 宏包,因此能够便捷地对中文论文进行排版,
    tikz 宏包使我能够绘出本文中的各种图,\hologo{BibTeX} 和 gbt7714 宏包能让我便捷地做出
    符合规范的参考文献。

    就计算机联锁,要感谢尚庆松老师。尚老师传授了的相关知识给我、为毕业设计提供指导意见并
    斧正本文,匡益实多。

    言虽如此,张睿不佞,弇陋不文,是非然否,不敢固也,遑论致谢?
    求早日付梓而所论谫疏,愧诸君相助也。
\end{arigatou}