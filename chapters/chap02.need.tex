\section{需求分析}
uroj旨在设计一款高性能、可扩展、高并发、通用性计算机联锁培训系统。
通用性在于我们需要一款不拘束于某个具体车站的计算机联锁培训系统。本系统需要拥有对任意一车站进行
联锁培训的能力,可扩展性在于需要让用户可以自己定义车站。用户可以在自定义的车站上进行联锁培训,
并且要求定义过程简单易懂,不需要用户输入多余的内容。

若要满足这一点,传统的联锁表必然不符合,联锁表相当于枚举了整个车站的设备,车站规模越大,
联锁表就越复杂,而且如果要求用户上传联锁表用于联锁逻辑,则还要上传站场图用于渲染,但
显然的,站场图和联锁表描述的都是同一种事物--车站。由于这些问题,uroj选择了图论算法
来驱动联锁逻辑。而用户只需要编辑上传一个文件就能描述整个车站:包括各种设备之间的耦合关系
还有车站的地理布置。并使用这个车站来创建自己的计算机联锁模拟环境,这就是本设计的目标。


高性能在于作为一款web应用,尽量缩短相应时间、提升硬件利用效率,减少冗余。
高并发在于作为一款web应用,通过设计保证系统能够同时并行处理尽量多的请求。