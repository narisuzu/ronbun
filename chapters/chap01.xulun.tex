\section{序论}
\subsection{研究背景与意义}
计算机联锁系统(CI)是综合运用计算机技术、网络通信技术和现代控制技术,
以电子信息传输方式集中操纵动力式道岔及色灯信号机等信号设备,从而实现控制车站的信号系统的功能的自动控制系统。
铁路车站是以建立进路的方式实现对列车、车列的运行控制。
计算机联锁系统通过计算机运算方式确保信号、道岔、进路间的相互关系正确,
满足各种车站、车场规模化运输作业的需要,能够保证行车安全,提高运输效率,改善劳动条件,
是车站信号设备控制系统的发展方向,是实现铁路现代化的重要基础之一。
联锁系统是铁路车站保证列车和车列正常和安全运行必不可少的核心基础设备。
而计算机联锁培训系统则是培训相关从业人员计算机联锁内容与相关操作的考试练习系统。
计算机联锁培训系统的目的在于:仿真模拟出现场的车站作业环境,能够达到练习、考试等培训功能,
通过练习功能、帮助学生深入学习计算机联锁,对计算机联锁的结构与操作工序逐渐的熟练,通过考试,
让教师能够对学生的水平有直观的判断与考察。提升计算机联锁教学的效率与品质。传统上,
计算机联锁培训通过视频教学、书本教学等培训手段,或者硬件模拟联锁逻辑实现,无法使仿真培训普及化以及并行化。
传统的计算机联锁教学相较于计算机联锁培训系统,更像是纸上谈兵,作为一门强实践性的技术,
计算机联锁培训系统能大大提高教育与学习的效率,已经是当今计算机联锁培训教育的首选教学工具\cite{intercnki}。
若能结合联锁逻辑的纯软件模拟和web技术,则能让计算机联锁系统不再掣肘于空间与时间。
只要在可存取的互联网环境下,则可以处处进行计算机联锁的仿真学习。降低学习成本,提升学习效率。
摆脱了硬件仿真,则教学系统可以几近于零成本地部署在各处的服务器上,降低了安装部署的成本。
因此计算机联锁逻辑结合网络技术的应用,在计算机联锁培训领域内是十分重要的发展方向。

\subsection{国内外研究发展现状}
\subsubsection{国外现状}
国外对于计算机仿真培训系统采用了很多的技术手段来实现,因涉及很多学科,所以起步较早。
发达国家在上世纪八十年代就提出了相关的概念,并在十年内快速发展并取得良好的实用。
例如:英国铁路部门采用了计算机网络、计算机模拟器、信号模拟器等技术手段进行搭建系统,使用于模拟和培训;
美国主要采用计算机仿真技术、多媒体技术、网络技术等组成实训系统。
日本采用力学仿真及数据运算的同时还使用多媒体闭路电视进行直观培训。\cite{intercnki}

\subsubsection{国内现状}

在国内,计算机联锁培训系统主要分为以下三种形式:纯实物仿真培训、沙盘模拟仿真系统\cite{电气化铁路信号计算机联锁实训系统研究与设计}、纯软件的计算机联锁仿真培训系统。

纯实物仿真培训采用真实的信号机、钢轨、转辙器、轨道电路、列车等轨道交通信号设备作为仿真的表示层,
采用真实的计算机联锁软件进行仿真培训,这种培训方式的优点在于,最贴近实际作业环境,
能最真实的反应出学生在真实站场中可能遇到的各种场景和问题,表示直观,能体现出实际站场中信号设备和计算机联锁系统的耦合逻辑。
其缺点在于,成本耗费巨大,硬件调试困难繁杂,越复杂的车站便要越大的空间才能部署,培训难以并行化,车站毫无扩展性,
严重依赖硬件作为表示层,使培训流程严重受限于硬件状态。比如南京铁道职业技术学院内设有两个车站,相去1公里。

沙盘模拟仿真系统采用真实的信号机、钢轨、转辙器、轨道电路、列车等轨道交通信号设备作为仿真的表示层,
采用真实的计算机联锁软件进行仿真培训,这种培训方式的优点在于,拥有实物模型,能够直观的看到整个车站实际的状态,
比较贴近实际作业环境,占地面积小,整个车站只需要一间房间即可容下,相较于纯实物仿真而言成本低廉,但拥有纯实物仿真培训的大多数优点,此外,
由于可以将多个车站构建于一个沙盘上,每一个车站都能同时进行一人的培训。缺点:构建沙盘需要一定的成本,需要硬件调试,培训流程一定程度掣肘于硬件状态
,囿于列车的数量和位置,车站之间耦合性强,一个学生的培训流程会被其他学生的培训流程所影响,进而影响整个培训系统的效率。
整个系统能承载的学生和车站种类有限,可扩展性较差,如果沙盘上部署的车站不同,则无法进行标准化考试,只可作为练习使用。
吕永宏使用了模拟沙盘、 小模型信号机、转辙器、轨道电路等硬件结合visual c++ 开发的教师机和学生机制作了微缩的实物仿真系统\cite{lyh}。

纯软件的计算机联锁仿真培训系统采用完全软件模拟的表示层和联锁逻辑构建培训系统进行计算力联锁的仿真培训,
这种模式的优点在于没有任何的硬件电路,可用性强,可以通过升级软件等方式获得其他车站,而联锁逻辑是独立的通用模块无需更动,扩展性强,易于维护。
只需要计算机即可完成部署,不依赖大面积空间,成本低廉。不依赖硬件状态,可以不受限于其他车站的状态,软件可以复制,便于分发,
同一个车站可以分发给多个学生进行练习,能够实现并行化考试与练习,可以任意组合车站,不囿于物理的车站连接关系。易于设定权限组分别提升教师权限,
不掣肘于列车数量,可以随意增开列车排列进路。方便在考试前设定题目和评分标准,使所有学生同时使用相同的车站完成相同的任务方便标准化的考试,
产生有参考价值的考试成绩。陈杰使用了web技术、仿真技术设计的计算机联锁培训系统实现了模拟联锁机硬件、专家和考试系统的设计实现方案\cite{cj}。
