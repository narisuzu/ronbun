%!ORDER = 4
\begin{problem}
下図の$LC$並列回路に電圧$\mscrE$を加えたとき, 電流計の指示が$0$となる周波数はいくら.
\begin{figure}[H]%使用figure环境
    \centering
    \begin{circuitikz}[american voltages]
        \draw (0,3) to[ammeter, o-] (3,3) to[short, -*] (3,2.25);
        \draw (0,0) to[short, o-] (3,0) to[short, -*] (3,0.75);
        \draw (2.25, 0.75) to[C=$C$] (2.25, 2.25) to (3.75, 2.25) to[L=$L$] (3.75, 0.75) to (2.25, 0.75);
        \draw (0,3) to[open, v=$\mscrE$] (0,0);
    \end{circuitikz}
    \caption{問題2の回路}%添加标题
\end{figure}
\end{problem}

\begin{solve}
    \pair{
        \begin{align*}
            Z = \frac{1}{\cj\omega C} // (\cj\omega L) = \frac{1}{\dfrac{1}{\omega L} - \omega C}\cj
        \end{align*}
    }{
        インピーダンス$Z$を求める
    }

    \pair{\begin{align*}
             & I = \frac{E}{Z} = E\left(\omega C - \dfrac{1}{\omega L}\right)\cj \\
             & |I| = E\left(\omega C - \dfrac{1}{\omega L}\right)
        \end{align*}
        であるから, $\omega C = \dfrac{1}{\omega L}$である限り, 電流計の指示が$0$, 即ち
        \begin{equation*}
            \omega = \frac{1}{\sqrt{LC}} \qedhere
        \end{equation*}
    }{
        電流$I$を求める
    }
\end{solve}
