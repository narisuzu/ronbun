%!ORDER = 2
\begin{problem}
下図の回路に於いて, インダクタンス$L_1$, $L_2$に電流は流れていないものとする.
今$t=0$でスイッチ$S$を閉じたとき$L_1$, $L_2$にはいくらの電流が流れるか.
ただし, キャパシタンス$C$の初期電荷を$Q$とする.
\begin{figure}[H]%使用figure环境
    \centering
    \begin{circuitikz}[american voltages]
        \draw (0,0) to[C=$C$, v<=$u(t)$, i=$i(t)$] (0,3) to[spst=$S$] (2,3) to (4,3) to[L=$L_2$, i=$i_2(t)$] (4,0) to (0,0);
        \draw (2,3) to[L=$L_1$,i=$i_1(t)$, *-*] (2,0);
    \end{circuitikz}
    \caption{問題2の回路}%添加标题
\end{figure}
\end{problem}

\begin{solve}
    \pair{
        \begin{equation}
            i = i_1 + i_2
        \end{equation}
        ただし
        \begin{align}
             & i = C\frac{\dif u}{\dif t}                                  \\
             & u = L_1\frac{\dif i_1}{\dif t} = L_2\frac{\dif i_2}{\dif t}
        \end{align}
        (3)式には(4),(5)を代入すると
        \begin{equation}
            C\frac{\dif^2 u}{\dif t^2} = \left(\frac{1}{L_1} + \frac{1}{L_2}\right)u
        \end{equation}
        ただし, $t = 0$のとき$u = \dfrac{Q}{C}$, 即ち,
        $$u(0) = \dfrac{Q}{C}$$
        (6)式を整理すると,
        \begin{equation}
            u''-\frac{L_1+L_2}{L_1L_2C}u = 0
        \end{equation}
        という微分方程式を得る.
    }{
        KCLとVCRで回路方程式を導出する
    }
    \pair{
        (7)式の特性方程式は
        $$p^2 - \frac{L_1+L_2}{L_1L_2C} = 0$$
        即ち
        \begin{equation}
            p = \pm\sqrt{\frac{L_1+L_2}{L_1L_2C}}
        \end{equation}
        だから,
        \begin{align}
             & u = A_1\ce^{\sqrt{\frac{L_1+L_2}{L_1L_2C}}t} + A_2\ce^{-\sqrt{\frac{L_1+L_2}{L_1L_2C}}t}                                             \\
             & i = C\sqrt{\frac{L_1+L_2}{L_1L_2C}}\left(A_1\ce^{\sqrt{\frac{L_1+L_2}{L_1L_2C}}t} - A_2\ce^{-\sqrt{\frac{L_1+L_2}{L_1L_2C}}t}\right)
        \end{align}
        $u(0) = \dfrac{Q}{C}$, $i(0) = 0$という初期条件は代入すれば,
        \begin{equation}
            \begin{cases}
                A_1 + A_2 = \dfrac{Q}{C} \\
                A_1 - A_2 = 0
            \end{cases}
        \end{equation}
        即ち,
        \begin{equation}
            A_1 = A_2 = \dfrac{Q}{2C}
        \end{equation}
        (9)式に代入しれば,
        $$u = \frac{Q}{2C}\left(\ce^{\sqrt{\frac{L_1+L_2}{L_1L_2C}}t} + \ce^{-\sqrt{\frac{L_1+L_2}{L_1L_2C}}t}\right)$$}{
        微分方程式を解く
    }

    \pair{
        (5)式より
        \begin{align*}
            i_1 & = \frac{1}{L_1}\int u \dif t + C                                                                                                               \\
                & = \frac{Q}{2CL_1\sqrt{\frac{L_1+L_2}{L_1L_2C}}}\left(\ce^{\sqrt{\frac{L_1+L_2}{L_1L_2C}}t} - \ce^{-\sqrt{\frac{L_1+L_2}{L_1L_2C}}t}\right) + C
        \end{align*}
        $i_1(0) = 0$であるから$C = 0$.
        \begin{equation*}
            i_1 = \frac{Q}{2CL_1\sqrt{\frac{L_1+L_2}{L_1L_2C}}}\left(\ce^{\sqrt{\frac{L_1+L_2}{L_1L_2C}}t} - \ce^{-\sqrt{\frac{L_1+L_2}{L_1L_2C}}t}\right)
        \end{equation*}
        同じように
        \begin{equation*}
            i_2 = \frac{Q}{2CL_2\sqrt{\frac{L_1+L_2}{L_1L_2C}}}\left(\ce^{\sqrt{\frac{L_1+L_2}{L_1L_2C}}t} - \ce^{-\sqrt{\frac{L_1+L_2}{L_1L_2C}}t}\right) \qedhere
        \end{equation*}
    }{
        $i_1$と$i_2$を求める
    }
\end{solve}
