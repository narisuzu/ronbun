\section{杂项}
本章记载一些无法纳入前文结构的一些内容
\subsection{时区}
本项目从数据库PostgreSQL、业务逻辑层到前端的表现层,一个需要重视的问题是时区。
若是单时区应用的话大可以直接在数据库中保存北京时间的字符串,但本案为了保证通用性
,考虑来自不同时区的用户,或者是部署于不同时区的数据库、服务器。因此在PostgreSQL中,
本例使用时间戳(timestamp)保存时间,时间戳是UTC1970年1月1日0时0分0秒起至现在的总秒数。
因此是一种没有时区的信息,或者说是协调世界时0时区的时间。

从数据库中取到的时间戳,在业务逻辑层中被表示为 chrono 库中的 NaiveDateTime,NaiveDateTime是
一种没有时区的日期时间字面量,其可以被简单地转换成任何时区的DateTime,因此我们会在
业务逻辑层中将0时区的时间戳转换成业务逻辑层服务器所在时区的时间,并呈递给表现层。表现层
显示哪种时区完全取决于其访问部署于何处的服务器。

\subsection{网关}
本案采用 Apollo 提供的Apollo Federation 功能和 ApolloGateway 配置网关
如下:

\begin{lstlisting}
const gateway = new ApolloGateway({
  serviceList: [
    { name: 'auth', url: 'http://localhost:8001' },
    { name: 'api', url: 'http://localhost:8002' },
  ],
});
\end{lstlisting}

Apollo Federation 可以将多个GraphQL 服务器统合成一个GraphQL服务器向外暴露
服务。例如上代码所示,使用位于:8001端口的auth服务和位于:8002端口的api服务配置网关。

\subsection{部署}
