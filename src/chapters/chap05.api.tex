\section{API服务}
API服务主要负责耦合data层和view层,上承用户的请求,下接数据库,从数据库读写数据并呈递给前端。
在本案中,API服务提供车站、实例、考试、用户和班级四个类型的服务
\subsection{车站}
在api层中,Station 数据是最完备最上游的车站静态数据,其直接来源于用户的输入。
车站模型定义为


\subsection{新建实例}
Instance是Station的实例,所以 Instance 需要 Station 数据进行初始化,另外为了创建Instance,
还需要一些必要的信息,譬如Instance类型,Instance支持三种类型:练习,考试和链,考试和链都需要一些描述该类型的详细内容,譬如考题与判分标准等,
再如Instance需要和用户交互,所以需要在创建实例时指定实例的用户,实例用户和实例创建者不一定相同,
在大多数练习场景中自然是相同的,但是在考试场景中,考试实例一般是由管理员(教师)创建给普通用户(学生)的。
因为本案是分布式架构,因此整个系统不一定只有一台 Executor,因此创建Instance时需要指定一个 Executor 以执行该Instance。
另外,用户创建实例是还需要为其指定标题与描述(可选)。

当新建实例之后,API会为实例生成唯一的UUID(Universally Unique Identifier),
UUID是用于计算机体系中以识别信息数目的一个128位标识符,
UUID根据标准方法生成,不依赖中央机构的注册和分配,
UUID具有唯一性,这与其他大多数编号方案不同。重复UUID码概率接近零,可以忽略不计。
因此UUID十分适合用在分布式系统数据表的主键。因为服务集群中即使有多个数据库、多个服务节点
也能保证某个实例的主键是世界上唯一的。

uroj使用PostgreSQL的gen\_random\_uuid函数生成版本4的UUID。

\subsection{用户}
\subsection{班级}