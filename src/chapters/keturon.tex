\begin{keturon}
    展望:目前为实现热插拔,没有在运行时使用网关,在表现层通过实例配置时指定的
    运行时创建对应运行时服务的Apollo客户端,从而实现在指定的运行时上运行实例。
    本案的原初设计是在api服务中集成实例调度器(Instance Scheduler),
    其具体职能是管理调度安排实例
    运行在哪个运行时上,还可以将多个运行时视为一种计算“资源”,
    将多个运行时服务池化称为运行时池,通过池中各个运行时的
    负载(运行实例数量、CPU负责、网络IO)等参数来决策新开始的实例应该由哪个运行时
    来运行。调度器通过tarpc(Google 开发的一个Rust专用的rpc专案)和各个运行时连接
    各个运行时作为tarpc服务器,调度器作为tarpc客户端。调度器在某个实例开始时间到时
    通过tarpc服务将新建实例配置传送给一个运行时以运行。
    但是十分遗憾的是,在开发完成调试程序时,发现tarpc要求异步库tokio的最低版本是1.0,
    而本案使用的web 异步框架
    actix-web(latest release)使用tokio版本是0.3,当时我寻求actix-web的替代品,但是
    本案的设计宗旨就是高性能,因为垂涎于actix-web的效率,实在不忍心将actix-web替换为
    别的web框架。因此便舍去了调度器方案。

    在此之后,我又尝试使用redis或者etcd等方案来实现,奈何时间过于紧迫,没添加一项新
    流程、新技术,都意味着阅读学习其使用文档说明、配置环境、处理和现有架构的耦合等等问题,
    要花费的时间不可估算(字面上的意思)。因此最终还是舍弃了使用redis或etcd的方案。
    随着actix-web的更新(actix-web的beta版已经使用tokio 1.0了)再将调度器加入api服务中
    就可以了。
\end{keturon}