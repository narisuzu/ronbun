\section{数据库架构}
本案选用 ProsgreSQL 作为数据库管理系统,PostgreSQL是开源的对象-关系数据库数据库管理系统,
在类似BSD许可与MIT许可的PostgreSQL许可下发行。

\subsection{表结构}
\subsubsection{classes}
\begin{lstlisting}[language = SQL]
CREATE TABLE classes(
    id         serial       primary key,
    class_name varchar(50)  unique not null
)
\end{lstlisting}
主键id为自增整数,class\_name为班级名称
\subsubsection{users}
\begin{lstlisting}[language = SQL]
CREATE TABLE users (
    id             varchar(30)     primary key,
    hash_pwd       varchar(60)     not null,
    email          text            not null,
    class_id       int  references classes(id) on delete set null,
    user_role      varchar(20)     not null,
    is_active      boolean         not null default 't',
    joined_at      timestamp       not null default now(),
    last_login_at  timestamp       default now()
)
\end{lstlisting}
解释:
\begin{itemize}
    \item 主键为id,类型为字符串,即用户自定义的用户id
    \item hash\_pwd 为加密后的用户密码
    \item email 为用户的电子邮箱地址
    \item class\_id 是表classes 的外键,表示用户所属的班级
    \item user\_role 表示用户角色
    \item is\_active 表示账户是否可用(未被禁用)
    \item joined\_at 和 last\_login\_at 默认是插入时的时间
\end{itemize}

\subsubsection{stations}
\begin{lstlisting}
CREATE TABLE stations (
    id          serial        primary key,
    title       varchar(250)  not null,
    description text,
    created_at  timestamp     not null default now(),
    updated_at  timestamp     not null default now(),
    draft       boolean       not null default 'f',
    author_id varchar(30) references users(id) on delete set null,
    yaml        text          not null
)
\end{lstlisting}
解释:
\begin{itemize}
    \item 主键为id, 自增整数
    \item title 为车站的标题
    \item description 为可空键,表示车站的备注
    \item draft 表示是否为草稿
    \item author\_id 是表 users 的外键,表示作者
    \item created\_at 和 updated\_at 默认是插入时的时间
    \item yaml 即为车站的描述文件内容
\end{itemize}

\subsubsection{executors}
\begin{lstlisting}
CREATE TABLE executors (
    id             serial        primary key,
    addr           varchar(50)   not null
)
\end{lstlisting}
addr是该执行器的地址。

\subsubsection{instances}
\begin{lstlisting}
CREATE TABLE instances (
    id      uuid    primary key default gen_random_uuid(),
    title   varchar(250)  not null,
    description text,
    created_at  timestamp     not null  default now(),
    creator varchar(30) references users(id) on delete set null,
    player  varchar(30)   not null  references users(id),    
    yaml        text          not null,
    curr_state  varchar(10)   not null,
    begin_at    timestamp     not null  default now(),
    executor_id int           not null  references executors(id),
    token       varchar(6)    not null
)
\end{lstlisting}
解释:
\begin{itemize}
    \item 主键为id, 类型是uuid
    \item title 为实例标题
    \item curr\_state 表示实例当前的状态
    \item creator 是表 users 的外键,表示创建者
    \item player 是表 users 的外键,表示实例的用户
    \item created\_at  默认是插入时的时间
    \item begin\_at 表示实例的开始时间
    \item executor\_id 表示该实例的运行时id,是executors表的外键
    \item yaml 即为车站的描述文件内容
    \item token 是游客令牌
\end{itemize}

\subsection{ORM}
本案采用ORM以提升开发效率,ORM 是一种程序设计技术,用于将数据库的记录映射到程序语言的对象中,
或者将对象映射到某个表中,其封装了CRUD的SQL语句操作,可以让开发者从表中直接读入一个对象。或者将
一个对象插入某个表。效果上说,它其实是创建了一个可在编程语言里使用的“虚拟对象数据库”。

在本案的uroj-db library 中,定义了程序的DAO层逻辑,封装了所有项目需要的数据库访问方法。以便供
Api, Auth, Executor 等服务复用。uroj采用diesel作为本案的ORM库,将DAO层的各种结构体定义
和上小结所定义的SQL表映射起来的,就是diesel client所生成的schema。

这里简单介绍一下 diesel 的使用步骤。首先需要定义migration,migration可以简单理解为创建和删除表
的sql文件。将上一节的表定义好后。使用diesel生成schema,schema是diesel使用
rust macro定义的一些字段。之后我们需要定义DAO层的struct,对于一个表一般而言需要两种
struct,一个是读取用一个是插入用,但需要derive diesel提供的相应的过程宏,这样就可以
将sql表和DAO 层 struct映射起来,再使用diesel提供的方法进行CRUD操作。